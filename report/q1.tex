\section{问题一的模型的建立和求解}
\subsection{问题一的描述与分析}

问题一考虑单个微基站的资源分配场景。该基站拥有50个资源块(Resource Block, RB),需要为三类网络切片——URLLC(高可靠低时延)、eMBB(增强移动宽带)和mMTC(大规模机器通信)进行资源分配,以最大化用户服务质量。这是一个静态资源分配优化问题,需要在满足资源约束的条件下,找到最优的资源块分配方案。

\subsection{预备工作}
为便于复现与对比,本问采用如下统一的数据、参数与约定:

\begin{itemize}
  \item 数据来源(附件1,单时刻数据):任务量 \texttt{q1\_任务流.csv}(单位:Mbit)、大规模衰减 \texttt{q1\_大规模衰减.csv}(单位:dB)、小规模瑞利衰减 \texttt{q1\_小规模瑞丽衰减.csv}(幅度 $|h|$)。按列名一一对应同名用户,忽略非数值列(如 \texttt{Time})。
  \item 系统与物理层参数:单小区、无同频干扰;发射功率 $p_{\mathrm{tx}}=30\,\mathrm{dBm}$;单RB带宽 $b=360\,\mathrm{kHz}$;噪声系数 $NF=7\,\mathrm{dB}$。白噪声按如下公式计算:
  \begin{equation}
    N_0\,(\mathrm{dBm}) = -174 + 10\log_{10}(i\,b) + NF
  \end{equation}
  并换算为mW。
  \item 资源占用粒度(同表1):URLLC/eMBB/mMTC 用户并发占用RB数分别为 $v_U=10$、$v_e=5$、$v_m=2$。为避免RB碎片化,约束切片RB分配满足 $n_U\bmod 10=0$、$n_e\bmod 5=0$、$n_m\bmod 2=0$,且 $n_U+n_e+n_m=50$。
  \item SLA与QoS:URLLC 时延SLA $L_U^{\mathrm{SLA}}=5\,\mathrm{ms}$、折扣因子 $\alpha=0.95$、惩罚 $M_U=5$;eMBB 速率SLA $r_e^{\mathrm{SLA}}=50\,\mathrm{Mbps}$、时延SLA $L_e^{\mathrm{SLA}}=100\,\mathrm{ms}$、惩罚 $M_e=3$;mMTC 时延SLA $L_m^{\mathrm{SLA}}=500\,\mathrm{ms}$、惩罚 $M_m=1$。
  \item 排队与评估口径:同一决策下允许在切片内“编号靠前优先”的串并行调度,后到用户可在前序用户完成后接续占用RB;URLLC/eMBB 以用户级QoS加和;mMTC 以满足SLA的接入比例计分(当期存在任务的mMTC用户数作分母)。
\end{itemize}
\subsection{模型建立}

\subsubsection{传输速率计算}

根据附录中的信号传输模型,用户$k$获得$i_k$个资源块时的接收功率为:

\begin{equation}
p_{\text{rx},k} = 10^{\frac{p_{\text{tx}} - \phi_k}{10}} \cdot |h_k|^2 \quad \text{(mW)}
\end{equation}
其中,$p_{\text{tx}}$为基站发射功率(dBm),$\phi_k$为大规模衰减(dB),$h_k$为小规模瑞利衰落系数,$p_{\text{rx},k}$为接收功率(mW)。

考虑噪声功率的影响,噪声功率谱密度为:

\begin{equation}
N_0 = -174 + 10\log_{10}(i_k \cdot b) + 7 \quad \text{(dBm)}
\end{equation}
其中,$i_k$为用户$k$占用的RB数量,$b$为单RB带宽(Hz),$-174$ dBm/Hz为热噪声谱密度,$7$ dB为噪声系数。

信干噪比(SINR)在无干扰情况下简化为信噪比(SNR):

\begin{equation}
\gamma_k = \frac{p_{\text{rx},k}}{10^{\frac{N_0}{10}}}
\end{equation}
其中,$N_0$以dBm计,$10^{\frac{N_0}{10}}$为噪声功率(mW)。

根据香农公式,用户$k$的传输速率为:

\begin{equation}
r_k = i_k \cdot b \cdot \log_2(1 + \gamma_k) \quad \text{(bps)}
\end{equation}
其中,$r_k$为传输速率(bps),$i_k$为RB数量,$b$为单RB带宽。

\subsubsection{服务质量评估函数}

根据附录中的用户服务质量定义,不同切片的QoS评估函数如下:

\textbf{(1) U切片(URLLC)}

用户$k$的传输时延为:
\begin{equation}
T_k = \frac{D_k \times 10^6}{r_k} \quad \text{(s)}
\end{equation}
总时延为:
\begin{equation}
L_k^{s} = Q_k + T_k, \quad s \in \{U, e, m\}
\end{equation}
其中,$D_k$为任务数据量(Mbit),$Q_k$为排队时延,$T_k$为传输时延。

服务质量函数为:
\begin{equation}
y_k^{U} = \begin{cases}
\alpha^{L_k^{U}} & \text{若 } L_k^{U} \leq L_{U}^{\text{SLA}} \\
-M_{U} & \text{若 } L_k^{U} > L_{U}^{\text{SLA}}
\end{cases}
\end{equation}
其中,$\alpha\in(0,1)$为效益折扣系数(本题取$\alpha=0.95$),$M_U$为U切片任务丢失惩罚系数,$L_U^{\text{SLA}}$为U切片时延SLA。

\textbf{(2) e切片(eMBB)}

e切片用户采用三段式QoS函数:
\begin{equation}
y_k^{e} = \begin{cases}
1 & \text{若 } r_k \geq r_{e}^{\text{SLA}} \text{ 且 } L_k^{e} \leq L_{e}^{\text{SLA}} \\
\frac{r_k}{r_{e}^{\text{SLA}}} & \text{若 } r_k < r_{e}^{\text{SLA}} \text{ 且 } L_k^{e} \leq L_{e}^{\text{SLA}} \\
-M_{e} & \text{若 } L_k^{e} > L_{e}^{\text{SLA}}
\end{cases}
\end{equation}
其中,$r_e^{\text{SLA}}$、$L_e^{\text{SLA}}$分别为e切片的速率与时延SLA,$M_e$为惩罚系数。

\textbf{(3) m切片(mMTC)}

m切片的QoS基于接入成功率(满足时延SLA时按接入比例计分):
\begin{equation}
y_k^{m} = \begin{cases}
\frac{\sum_{i \in \mathcal{U}_{m}} c_i'}{\sum_{i \in \mathcal{U}_{m}} c_i} & \text{若 } L_k^{m} \le L_{m}^{\text{SLA}} \\
-M_{m} & \text{若 } L_k^{m} > L_{m}^{\text{SLA}}
\end{cases}
\end{equation}
其中,$\mathcal{U}_m$为m切片用户集合,$c_i$表示是否有任务需求,$c_i'$表示是否成功接入,$L_k^{m}$为用户$k$的总时延(同上定义),$M_m$为惩罚系数。

\subsubsection{优化模型}

基于上述分析,建立如下优化模型:

\begin{equation}
\begin{aligned}
\max_{n_U, n_e, n_m} \quad & Q = \sum_{s \in \mathcal{S}} \sum_{k \in \mathcal{U}_s} y_k^s \\
\text{s.t.} \quad & \begin{cases}
 n_U + n_e + n_m \leq N \\
 n_s \geq 0, \quad \forall s \in \mathcal{S} \\
 n_s \in \mathbb{Z}, \quad \forall s \in \mathcal{S}
 \end{cases}
 \end{aligned}
 \end{equation}
其中,$n_U, n_e, n_m$分别为分配给U、e、m切片的RB个数,$\mathcal{S}=\{U,e,m\}$,$\mathcal{U}_s$为切片$s$的用户集合;$y_k^{U}$、$y_k^{e}$为用户级QoS得分,$y^{m}$为m切片的聚合式接入比例得分。

其中,第一个约束为资源块总数限制,第二个约束为非负约束,第三个约束为整数约束。
\subsection{模型求解}
为求解最优切片RB分配与同周期排队调度,采用“离散枚举 + 切片内串并行调度”的流程:

\begin{enumerate}
  \item 切片RB枚举:枚举 $R_U\in\{0,10,\dots,50\}$、$R_e\in\{0,5,\dots,50-R_U\}$,令 $R_m=50-R_U-R_e$,若 $R_m<0$ 或 $R_m\bmod 2\ne 0$ 则跳过。各切片并发容量 $\mathrm{cap}_U=\lfloor R_U/10\rfloor$、$\mathrm{cap}_e=\lfloor R_e/5\rfloor$、$\mathrm{cap}_m=\lfloor R_m/2\rfloor$。
  \item 切片内调度:对切片 $s\in\{U,e,m\}$,所有用户按“编号靠前优先”的固定顺序进入队列。以固定占用 $v_s$ 计算其独占传输时延 $T_k$ 与速率 $r_k$,并采用 $\mathrm{cap}_s$ 台并行服务器的无抢占调度:先填满并行槽位,之后每当最早完成的会话释放,队列头用户即时接续,占用直至完成。由此得到等待 $Q_k$ 与总时延 $L_k=Q_k+T_k$。
  \item QoS 计算与聚合:对URLLC与eMBB逐用户按 $L_k$(及 $r_k$)代入其QoS定义计分并求和;对mMTC统计满足 $L_k\le L_m^{\mathrm{SLA}}$ 的用户数占比作为 $y^{m}$。
  \item 目标计算与最优解提取:
  \[
    Q(R_U,R_e,R_m)=\sum_{k\in\mathcal U_U} y_k^U + \sum_{k\in\mathcal U_e} y_k^e + y^{m}
  \]
  取 $Q$ 最大的切片RB分配与相应调度结果作为最优解。
\end{enumerate}

该方法在分配层面保持小规模枚举,调度层面采用线性对数复杂度的最早完成时间堆维护,整体复杂度低、可复现,并严格符合题目“编号靠前优先、可在同一决策中依次占用RB”的处理规则。

\subsection{求解结果}

基于附件1数据与上述流程,得到允许排队后的最优分配与接入如下(发射功率固定为 $30\,\mathrm{dBm}$):

\begin{itemize}
  \item 最优切片RB分配:$R_U=20,\ R_e=10,\ R_m=20$(合计50)。
  \item 接入选择:URLLC 接入 \{U1, U2\};eMBB 接入 \{e1, e2, e3, e4\};mMTC 全部接入 \{m1\textasciitilde m10\}。
  \item 关键指标:mMTC 接入比例 $y^{m}=1.0000$;URLLC QoS 合计 $\sum y^{U}=1.9870$;eMBB QoS 合计 $\sum y^{e}=3.7953$;目标函数 $Q=6.7823$。
\end{itemize}

为便于阅览,给出最优方案下各用户关键指标表。

\begin{table}[htbp]
  \centering
  \small
  \caption{URLLC 用户关键指标(最优分配 $R_U=20,\ R_e=10,\ R_m=20$)}
  \begin{tabular}{lccc}
    \hline
    用户 & r(Mbps) & L(ms) & QoS \\
    \hline
    U1 & 71.028 & 0.145 & 0.9926 \\
    U2 & 100.783 & 0.110 & 0.9944 \\
    \hline
  \end{tabular}
\end{table}

\begin{table}[htbp]
  \centering
  \small
  \caption{eMBB 用户关键指标(最优分配 $R_U=20,\ R_e=10,\ R_m=20$)}
  \begin{tabular}{lccc}
    \hline
    用户 & r(Mbps) & L(ms) & QoS \\
    \hline
    e1 & 54.217 & 2.017 & 1.0000 \\
    e2 & 51.686 & 2.182 & 1.0000 \\
    e3 & 39.766 & 4.546 & 0.7953 \\
    e4 & 57.878 & 5.605 & 1.0000 \\
    \hline
  \end{tabular}
\end{table}

\begin{table}[htbp]
  \centering
  \small
  \caption{mMTC 用户关键指标(最优分配 $R_U=20,\ R_e=10,\ R_m=20$)}
  \begin{tabular}{lccc}
    \hline
    用户 & r(Mbps) & L(ms) & 是否满足SLA \\
    \hline
    m1 & 19.524 & 0.623 & 是 \\
    m2 & 19.607 & 0.639 & 是 \\
    m3 & 17.219 & 0.747 & 是 \\
    m4 & 21.306 & 0.580 & 是 \\
    m5 & 20.234 & 0.607 & 是 \\
    m6 & 23.069 & 0.596 & 是 \\
    m7 & 21.489 & 0.610 & 是 \\
    m8 & 19.334 & 0.709 & 是 \\
    m9 & 17.815 & 0.713 & 是 \\
    m10 & 19.062 & 0.672 & 是 \\
    \hline
  \end{tabular}
\end{table}
