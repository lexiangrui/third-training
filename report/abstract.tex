\begin{abstract}
本文针对无线网络切片资源管理的一系列复杂优化问题,设计并实现了一套从静态到动态、从同构到异构、从性能优先到兼顾能耗的渐进式求解方案。

\textbf{对于问题一,} 我们针对单基站静态场景,建立了服务质量(QoS)最大化模型。通过对所有可能的资源块(RB)分配方案进行穷举搜索,得到了全局最优解:为URLLC、eMBB、mMTC切片分别分配20、10、20个RB时,可达到最大QoS得分15.78。此结果为后续更复杂的问题提供了性能基准。

\textbf{对于问题二,} 面对动态任务到达和时变信道,我们引入了\textbf{模型预测控制(MPC)}框架。将总时长划分为10个100ms的决策窗口,在每个窗口开始时,根据当前系统状态(如用户队列)进行枚举寻优。该方法有效应对了系统的动态性,实现了337.42的累计QoS得分,验证了MPC框架处理时变问题的有效性。

\textbf{对于问题三,} 在存在同频干扰的多微基站场景下,决策维度急剧增加。我们建立了用户接入、RB分配和功率控制的联合优化模型,并提出一种\textbf{MPC结合遗传算法(GA)}的两层求解框架。外层MPC负责时域滚动,内层GA通过精心设计的混合编码方案,高效求解每个窗口内非凸、高维的静态资源分配问题,最终在有效抑制干扰的同时,获得了853.37的累计QoS。

\textbf{对于问题四,} 针对宏基站(MBS)与微基站(SBS)共存的异构网络,我们对MPC+GA框架进行了扩展。模型引入了MBS与SBS的资源和功率异构性,以及用户只能接入MBS或最近SBS的约束。求解结果显示出智能的\textbf{网络功能协同策略}:无干扰、资源丰富的MBS主要承载海量mMTC连接,而靠近用户的SBS则重点保障URLLC等高性能业务,最终将累计QoS提升至1041.28。

\textbf{对于问题五,} 为在保障QoS的同时最小化网络能耗,我们设计了一种新颖的\textbf{两阶段优化算法}。在MPC框架的每个窗口内,第一阶段采用GA优化发射功率以最小化能耗;第二阶段在给定功率下,通过枚举优化RB分配以最大化QoS。该分层解耦策略成功平衡了性能与能耗的矛盾,在将总能耗控制在183.68焦耳的同时,取得了381.17的QoS得分。

综上,本文通过对一系列问题的层层递进的建模与求解,系统地展示了如何运用枚举、MPC、遗传算法及分层优化等方法,解决不同复杂度下的网络切片资源管理难题,为设计高效、智能的无线资源分配方案提供了有价值的思路与验证。

\keywords{网络切片\quad 资源管理\quad 服务质量(QoS)\quad 模型预测控制(MPC)\quad 遗传算法(GA)\quad 异构网络\quad 能源效率}
\end{abstract}