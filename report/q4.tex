\section{问题四的模型的建立和求解}
\subsection{问题四的描述与分析}

本问引入宏基站(Macro BS, 记作 MBS)与多个微基站(Small BS, 记作 SBS)的异构蜂窝网络:MBS 具备更充裕的频谱资源且覆盖广;SBS 负责边缘热点的增强覆盖。题设指出 MBS 与所有 SBS 采用\textbf{不重叠频谱},因此\textbf{跨层无干扰};但各 SBS 之间\textbf{同频复用},存在相互干扰。系统每 $100$ms 进行一次联合决策,周期内以 $1$ms 进行仿真演化。与第三问相比,第四问新增了跨层接入模式决策:\textbf{每个用户要么接入最近的 SBS,要么接入 MBS}(SBS 的选择被限定为“最近微站”)。

资源与功率设定:MBS 拥有 $100$ 个 RB,功率范围 $[10,40]$ dBm;每个 SBS 拥有 $50$ 个 RB,功率范围 $[10,30]$ dBm。三类切片的 RB 占用粒度与 SLA 约束与前几问一致(URLLC/eMBB/mMTC 分别占用 $10/5/2$ 个 RB,并满足表中速率/时延/SLA/惩罚系数约束)。附件四提供了 $1000$ms 窗口内各用户任务到达与位置、MBS 与各 SBS 的大/小规模衰落时序数据,支撑逐毫秒的链路与队列仿真。

本问是一个“跨层接入 + 多站切片 + 切片级功率控制 + 队列与 SLA 约束 + SBS 间干扰耦合”的时变混合整数非凸优化问题(MINLP)。为保持可计算性,下文给出一致化的符号体系、精确的数学建模与可操作的分层求解策略。

\subsection{预备工作}

\subsubsection{集合、索引与参数}

\begin{itemize}
  \item 基站集合:$\bar{\mathcal{N}}=\{0\}\cup\mathcal{N}$,其中 $0$ 表示 MBS,$\mathcal{N}=\{1,2,\dots,N_s\}$ 表示 SBS 集合(附件四为三站示例:$N_s=3$)。
  \item 切片集合:$\mathcal{S}=\{U,e,m\}$,分别对应 URLLC/eMBB/mMTC。
  \item 用户集合:$\mathcal{K}=\mathcal{K}_U\cup\mathcal{K}_e\cup\mathcal{K}_m$。
  \item 决策时刻集合:$\mathcal{T}=\{0,100,\dots,900\}$(单位 ms);窗口内细粒度时刻集合:$\mathcal{F}(t)=\{t,t+1,\dots,t+99\}$。
  \item RB 总数:$R_0=100$(MBS),$R_n=50$($n\in\mathcal{N}$,SBS)。单 RB 带宽 $b=360\,\mathrm{kHz}$,噪声系数 $NF=7\,\mathrm{dB}$。
  \item 切片占用粒度:$i_U=10,\ i_e=5,\ i_m=2$(每个\textbf{并发}用户占用的 RB 数)。
  \item 发射功率范围:$p_{0,s}(t)\in[10,40]$ dBm(MBS),$p_{n,s}(t)\in[10,30]$ dBm(SBS,$n\in\mathcal{N}$),为“基站 $n$ 在窗口 $t$ 对切片 $s$ 的统一每 RB 功率”。
  \item 候选接入集合:每个用户 $k$ 的候选基站集合限定为 $\{0,\operatorname*{argmin}_{n\in\mathcal{N}}\Phi_{n,k}\}$,即“宏站或最近微站”。其中 $\Phi_{n,k}$ 可取窗口平均的路径损耗或地理最近(由附件位置/损耗数据导出)。
\end{itemize}

附件四提供 $\phi_{n,k}(\tau)$(dB, 大规模损耗)、$h_{n,k}(\tau)$(瑞利小尺度)与任务到达序列 $D_k(\tau)$。决策窗口内按 $1$ms 时间步精确计算链路与服务过程。

\subsection{模型建立}

\subsubsection{信道与干扰模型}

若用户 $k\in\mathcal{K}_s$ 在窗口 $t$ 由基站 $n\in\bar{\mathcal{N}}$ 服务且被分配切片 $s$ 的并发资源槽,则 $\tau\in\mathcal{F}(t)$ 时刻接收功率(mW)为
\begin{equation}
 p_{\mathrm{rx},n\to k}(\tau)=10^{\frac{p_{n,s}(t)-\phi_{n,k}(\tau)}{10}}\cdot |h_{n,k}(\tau)|^2.
\end{equation}
噪声功率与占用 RB 数 $i_s$ 成正比,换算为线性功率(mW):
\begin{equation}
 N_0(i_s)=10^{\frac{-174+10\log_{10}(i_s\cdot b)+NF-30}{10}}.
\end{equation}
SBS 层采用同频复用,\textbf{仅}当不同 SBS 在同一 RB 索引上并发传输时产生互扰。为此,每个基站在窗口 $t$ 内将其 RB 在频域上按切片连续划分且顺序固定(例如 U-e-m),跨站的相同 RB 索引构成同信道。于是
\begin{equation}
 \gamma_k(\tau)=\frac{p_{\mathrm{rx},n\to k}(\tau)}{\underbrace{\sum\limits_{u\in\mathcal{N},\ u\neq n} I_{u\to k}(\tau)}_{\text{仅当 }n\in\mathcal{N}\text{ 时存在}}+N_0(i_s)},\quad s\in\mathcal{S},
\end{equation}
其中 $I_{u\to k}(\tau)$ 表示来自他站 $u$、在与 $k$ 所占 RB 索引重叠的切片 RB 上的干扰功率,按与上式相同的接收功率表达(由 $p_{u,s'}(t),\phi_{u,k}(\tau),h_{u,k}(\tau)$ 决定)。\textbf{对于 MBS($n=0$),跨层无干扰},其分母只含噪声项。基于香农公式,瞬时速率为
\begin{equation}
 r_k(\tau)=i_s\cdot b\cdot \log_2\big(1+\gamma_k(\tau)\big)\quad(\mathrm{bps}).
\end{equation}

\subsubsection{任务到达、队列与时延}

令 $D_k(\tau)\ge 0$ 表示 $1$ms 时刻 $\tau$ 到达用户 $k$ 的任务数据量(Mbit,来自 taskflow 数据)。任务按 FIFO 服务。记窗口起点的队列为 $Q_k(t)$。若窗口 $t$ 内被服务的时隙集合为 $\mathcal{U}_k(t)\subseteq\mathcal{F}(t)$,则窗口内可传输的数据量为
\begin{equation}
 S_k(t)=\sum_{\tau\in\mathcal{U}_k(t)} r_k(\tau)\cdot 10^{-6}\cdot 10^{-3}\quad(\mathrm{Mbit}).
\end{equation}
窗口结束时队列更新为
\begin{equation}
 Q_k(t+100)=\max\Big\{0,\ Q_k(t)+\sum_{\tau\in\mathcal{F}(t)} D_k(\tau)-S_k(t)\Big\}.
\end{equation}
对到达于时刻 $\tau$ 的任务,记开始服务时刻为 $t_{\text{start}}$,其排队时延 $Q_{k,\tau}=t_{\text{start}}-\tau$,传输时延 $T_{k,\tau}=\dfrac{D_{k,\tau}\cdot 10^6}{\bar{r}_k(t_{\text{start}})}$($\bar{r}$ 为窗口内代表速率),总时延 $L^{s}_{k,\tau}=Q_{k,\tau}+T_{k,\tau}$。

\subsubsection{接入与调度规则}

\begin{itemize}
  \item 接入限制:仅允许 $a_{0,k}(t)$ 与 $a_{n^*(k),k}(t)$ 之一为 $1$,其中 $n^*(k)$ 为 $k$ 的最近微站;其余基站的接入指示为 $0$。
  \item 切片并发容量:每个 $(n,s)$ 的并发容量为 $C_{n,s}(t)=\big\lfloor x_{n,s}(t)/i_s\big\rfloor$,窗口内采用“编号靠前优先”的串-并行调度:每个 $(n,s)$ 最多同时服务 $C_{n,s}(t)$ 个队头任务,任务完成即释放并补位。URLLC 在切片内可按紧迫度(距 SLA 的剩余时限)优先。
\end{itemize}

\subsubsection{QoS 评估函数}

与前两问一致,定义任务级 QoS:
\begin{equation}
 y^{U}_{k,\tau}=\begin{cases}
 \alpha^{L^{U}_{k,\tau}} & L^{U}_{k,\tau}\le L^{\text{SLA}}_{U}\\
 -M_U & \text{否则}
 \end{cases},\quad
 y^{e}_{k,\tau}=\begin{cases}
 1 & r_{k}(\tau_\text{srv})\ge r^{\text{SLA}}_{e}\ \&\ L^{e}_{k,\tau}\le L^{\text{SLA}}_{e}\\
 \dfrac{r_{k}(\tau_\text{srv})}{r^{\text{SLA}}_{e}} & r_{k}(\tau_\text{srv})< r^{\text{SLA}}_{e}\ \&\ L^{e}_{k,\tau}\le L^{\text{SLA}}_{e}\\
 -M_e & \text{否则}
 \end{cases}
\end{equation}
\begin{equation}
 y^{m}_{k,\tau}=\begin{cases}
 \dfrac{\sum\limits_{i\in\mathcal{K}_m}c'_i}{\sum\limits_{i\in\mathcal{K}_m}c_i} & L^{\,m}_{k,\tau}\le L^{\text{SLA}}_{m}\\
 -M_m & \text{否则}
 \end{cases}
\end{equation}
参数取值:$\alpha=0.95$,$r^{\text{SLA}}_e=50\,\mathrm{Mbps}$,$L^{\text{SLA}}_{U}=5\,\mathrm{ms}$,$L^{\text{SLA}}_{e}=100\,\mathrm{ms}$,$L^{\text{SLA}}_{m}=500\,\mathrm{ms}$,$M_U=5, M_e=3, M_m=1$。

\subsubsection{决策变量与优化模型}

决策变量:
\begin{itemize}
  \item RB 切片分配:$x_{n,s}(t)\in\mathbb{Z}_{\ge 0}$,$\sum\limits_{s\in\mathcal{S}} x_{n,s}(t)=R_n$,且 $x_{n,U}\bmod 10=0$、$x_{n,e}\bmod 5=0$、$x_{n,m}\bmod 2=0$。
  \item 发射功率:$p_{0,s}(t)\in[10,40]$ dBm,$p_{n,s}(t)\in[10,30]$ dBm($n\in\mathcal{N}$)。
  \item 接入关联:$a_{n,k}(t)\in\{0,1\}$,$\sum\limits_{n\in\bar{\mathcal{N}}} a_{n,k}(t)\le 1$,且 $a_{n,k}(t)=0$ 若 $n\notin\{0,n^*(k)\}$。
\end{itemize}

跨十个窗口的总体优化模型:
\begin{equation}
\begin{aligned}
\max\limits_{\{x,p,a\}}\quad & Q_{\text{total}}=\sum_{t\in\mathcal{T}}\Bigg[\sum_{k\in\mathcal{K}_U}\sum_{\tau\in\mathcal{A}_k(t)} y^{U}_{k,\tau}+\sum_{k\in\mathcal{K}_e}\sum_{\tau\in\mathcal{A}_k(t)} y^{e}_{k,\tau}+\sum_{k\in\mathcal{K}_m}\sum_{\tau\in\mathcal{A}_k(t)} y^{m}_{k,\tau}\Bigg] \\
\text{s.t.}\quad & \sum_{s\in\mathcal{S}} x_{n,s}(t)=R_n,\ x_{n,U}(t)\bmod 10=0,\ x_{n,e}(t)\bmod 5=0,\ x_{n,m}(t)\bmod 2=0,\\
& x_{n,s}(t)\in\mathbb{Z}_{\ge 0},\ \forall n\in\bar{\mathcal{N}},s\in\mathcal{S},t\in\mathcal{T},\\
& 10\le p_{0,s}(t)\le 40,\ 10\le p_{n,s}(t)\le 30\ (n\in\mathcal{N}),\\
& \sum_{n\in\bar{\mathcal{N}}} a_{n,k}(t)\le 1,\ a_{n,k}(t)=0\ \text{若 }n\notin\{0,n^*(k)\},\\
& r_k(\tau),\ \gamma_k(\tau)\ \text{由}\ (x,p,a)\ \text{与}\ (\phi,h)\ \text{及调度生成},\\
& Q_k(t+100)=\max\Big\{0,\ Q_k(t)+\sum_{\tau\in\mathcal{F}(t)} D_k(\tau)-S_k(t)\Big\},\ \forall k,t.
\end{aligned}
\end{equation}
其中 $\mathcal{A}_k(t)\subseteq\mathcal{F}(t)$ 为窗口 $t$ 内属于用户 $k$ 且在 SLA 内完成的任务到达时刻集合。该模型体现了“跨层接入选择 + 多站切片 + 切片级功率 + SBS 间互扰 + 任务队列”的耦合,属于时变 MINLP。

\subsection{模型求解}

直接全局求解不可行。我们采用“\textbf{MPC(单步前瞻)+ 分层分解 + 迭代最优响应}”的实用策略,在每个决策窗口内求解当前 $100$ms 的近似最优策略、并将末状态传递给下一窗口。

\paragraph{总体流程(逐窗口)}
对每个窗口 $t\in\mathcal{T}$:
\begin{enumerate}
  \item 状态采样:读取 $\tau\in\mathcal{F}(t)$ 的 $\{\phi, h\}$ 与任务到达 $D(\tau)$,获取上窗末的队列 $Q(t)$ 与上一窗的$(x,p,a)$ 作为初始参考。
  \item 关联初始化(跨层接入初判):对每个用户 $k$,在“\,MBS 最小功率/当前 $x$ 近似\,”与“\,最近 SBS 最小功率/当前 $x$ 近似\,”下分别估计\,URLLC/eMBB\,的 SLA 可达性与 mMTC 的接入概率,择优设置 $a^{(0)}_{n,k}(t)$;若两者均不可达,对 eMBB/URLLC 选择更高速率一侧,对 mMTC 选择负载更低者。
  \item 内层协调(给定 $a$,按基站分解 $x,p$):
  \begin{itemize}
    \item MBS 子问题(无互扰,独立):对 $x_{0,\cdot}(t)$ 枚举(满足 $R_0=100$ 与粒度约束),对 $p_{0,\cdot}(t)$ 采用粗网格搜索(如 $\{10,15,20,25,30,35,40\}$ dBm),逐个方案以 $1$ms 仿真评估窗口 QoS,取最优。
    \item SBS 子问题(存在互扰,耦合):采用\textbf{交替最优响应}(Block Coordinate Descent, BCD)。初始化 $x_{n,\cdot}(t),p_{n,\cdot}(t)$(可承接上窗或均匀切分)。循环若干轮:依次固定其余站的 $(x,p)$,对当前站 $n$ 在“$x$ 枚举 + $p$ 粗网格”下,仿真评估并更新最优;直至收敛或达到轮次上限(如 3 轮)。
  \end{itemize}
  \item 关联细化(在已得 $(x,p)$ 下微调 $a$):对每个用户在其候选集合内(MBS 与最近 SBS)尝试切换,比较 QoS 边际收益与被选基站切片的并发占用影响(若新侧切片已满并发,施加惩罚),采用贪心或少量轮换匹配(如至多交换 5% 用户)提高总 QoS。
  \item 输出窗口解:得到 $\{x_{n,s}(t),p_{n,s}(t),a_{n,k}(t)\}$,并以仿真末状态更新 $Q(t+100)$,进入下一窗口。
\end{enumerate}

\paragraph{复杂度与实现要点}
\begin{itemize}
  \item \textbf{枚举规模控制}:$x_{n,\cdot}(t)$ 的可行组合数有限(MBS 以 10/5/2 粒度划分 100RB,SBS 以 10/5/2 粒度划分 50RB),$p$ 采用 5--7 点粗网格。MBS 独立求解;SBS 经 BCD 将“同时组合爆炸”转化为“按站逐个改进”。
  \item \textbf{并发与调度一致性}:切片并发 $C_{n,s}(t)$ 由 $x$ 决定,窗口内以 1ms 步长执行“编号靠前优先 + 任务完成即补位”的调度,精确记录每个任务的 $(\tau, t_{\text{start}})$ 与完成时刻,严格计算 $L$ 与 QoS。
  \item \textbf{跨层隔离简化}:MBS 与 SBS 频谱隔离,使 MBS 子问题无互扰,且与 SBS 子问题仅通过 $a$(接入关联)耦合,利于分层求解与迭代收敛。
  \item \textbf{热启动}:建议采用“承接上窗口解”作为初值,既加速收敛又提升稳定性;在业务态势缓变时尤为有效。
\end{itemize}

\subsection{结果分析}

在不改变数据与评测口径的前提下,上述“MPC + 分层 + BCD + 细化接入”的策略具备如下预期:
\begin{itemize}
  \item \textbf{跨层负载分担}:当最近 SBS 对 eMBB/URLLC 的速率或时延不达标时,MBS 作为“高能小区”吸纳热点或边缘弱场景用户,保障 SLA;反之 mMTC 与近端 URLLC 倾向留在 SBS,提高频谱复用与连接密度。
  \item \textbf{SBS 间干扰自洽}:通过 BCD 在切片粒度对 RB 区段与功率进行协调,降低跨站同索引 RB 的重叠度或在易干扰切片上选择更保守的功率,提升总体 SINR 与 QoS。
  \item \textbf{时序一致性}:窗口内逐毫秒仿真与队列更新确保排队/传输时延与 mMTC 接入率的真实反映;结合上一窗热启动,决策轨迹平稳演化。
\end{itemize}

综上,本问的数学模型明确刻画了跨层接入、切片化 RB 分配、切片级功率控制与 SBS 干扰耦合,并以 MPC 框架下的分层-迭代最优响应给出可计算的近似最优解法,能在附件四提供的数据尺度上稳定运行并获得高 QoS 的资源调度策略。
