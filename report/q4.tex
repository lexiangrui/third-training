\section{问题四的模型的建立和求解}
\subsection{问题四的描述与分析}

本问引入宏基站(Macro BS, 记作 MBS)与多个微基站(Small BS, 记作 SBS)的异构蜂窝网络:MBS 具备更充裕的频谱资源且覆盖广;SBS 负责边缘热点的增强覆盖。题设指出 MBS 与所有 SBS 采用\textbf{不重叠频谱},因此\textbf{跨层无干扰};但各 SBS 之间\textbf{同频复用},存在相互干扰。系统每 $100$ms 进行一次联合决策,周期内以 $1$ms 进行仿真演化。与第三问相比,第四问新增了跨层接入模式决策:\textbf{每个用户要么接入最近的 SBS,要么接入 MBS}(SBS 的选择被限定为“最近微站”)。

资源与功率设定:MBS 拥有 $100$ 个 RB,功率范围 $[10,40]$ dBm;每个 SBS 拥有 $50$ 个 RB,功率范围 $[10,30]$ dBm。三类切片的 RB 占用粒度与 SLA 约束与前几问一致(URLLC/eMBB/mMTC 分别占用 $10/5/2$ 个 RB,并满足表中速率/时延/SLA/惩罚系数约束)。附件四提供了 $1000$ms 窗口内各用户任务到达与位置、MBS 与各 SBS 的大/小规模衰落时序数据,支撑逐毫秒的链路与队列仿真。

本问是一个“跨层接入 + 多站切片 + 切片级功率控制 + 队列与 SLA 约束 + SBS 间干扰耦合”的时变混合整数非凸优化问题(MINLP)。为保持可计算性,下文给出一致化的符号体系、精确的数学建模与可操作的分层求解策略。

\subsection{预备工作}

\subsubsection{集合、索引与参数}

\begin{itemize}
  \item 基站集合:$\bar{\mathcal{N}}=\{0\}\cup\mathcal{N}$,其中 $0$ 表示 MBS,$\mathcal{N}=\{1,2,\dots,N_s\}$ 表示 SBS 集合(附件四为三站示例:$N_s=3$)。
  \item 切片集合:$\mathcal{S}=\{U,e,m\}$,分别对应 URLLC/eMBB/mMTC。
  \item 用户集合:$\mathcal{K}=\mathcal{K}_U\cup\mathcal{K}_e\cup\mathcal{K}_m$。
  \item 决策时刻集合:$\mathcal{T}=\{0,100,\dots,900\}$(单位 ms);窗口内细粒度时刻集合:$\mathcal{F}(t)=\{t,t+1,\dots,t+99\}$。
  \item RB 总数:$R_0=100$(MBS),$R_n=50$($n\in\mathcal{N}$,SBS)。单 RB 带宽 $b=360\,\mathrm{kHz}$,噪声系数 $NF=7\,\mathrm{dB}$。
  \item 切片占用粒度:$i_U=10,\ i_e=5,\ i_m=2$(每个\textbf{并发}用户占用的 RB 数)。
  \item 发射功率范围:$p_{0,s}(t)\in[10,40]$ dBm(MBS),$p_{n,s}(t)\in[10,30]$ dBm(SBS,$n\in\mathcal{N}$),为“基站 $n$ 在窗口 $t$ 对切片 $s$ 的统一每 RB 功率”。
\end{itemize}

附件四提供 $\phi_{n,k}(\tau)$(dB, 大规模损耗)、$h_{n,k}(\tau)$(瑞利小尺度)与任务到达序列 $D_k(\tau)$。决策窗口内按 $1$ms 时间步精确计算链路与服务过程。

\subsection{模型建立}

\subsubsection{信道与干扰模型}

本问的信道模型核心要素与前文一致,用户的瞬时接收功率 $p_{\mathrm{rx},n\to k}(\tau)$ 与热噪声功率 $N_0(i_s)$ 的计算公式保持不变。我们在此基础上,重点阐述第四问独特的异构网络干扰模型。

根据题目设定,宏基站(MBS, $n=0$)与所有微基站(SBS, $n \in \mathcal{N}$)工作在不同频段,因此它们之间不存在跨层干扰。干扰仅存在于同频部署的各个 SBS 之间。用户的信干噪比(SINR)$\gamma_k(\tau)$ 因此取决于其所接入的基站类型:
\begin{itemize}
    \item \textbf{当用户接入 MBS ($n=0$) 时},不存在干扰,其接收质量由信噪比(SNR)决定:
    \begin{equation}
        \gamma_k(\tau) = \frac{p_{\mathrm{rx},0\to k}(\tau)}{N_0(i_s)}, \quad \text{若 } a_{0,k}(t)=1
    \end{equation}

    \item \textbf{当用户接入 SBS ($n \in \mathcal{N}$) 时},会受到来自其他 SBS 的同频干扰。干扰功率 $I_{u\to k}(\tau)$ 来自于其他 SBS $u$ ($u \in \mathcal{N}, u \neq n$) 在相同 RB 上的发射。此时,信干噪比为:
    \begin{equation}
        \gamma_k(\tau) = \frac{p_{\mathrm{rx},n\to k}(\tau)}{\sum_{u \in \mathcal{N}, u \neq n} I_{u\to k}(\tau) + N_0(i_s)}, \quad \text{若 } a_{n,k}(t)=1, n \in \mathcal{N}
    \end{equation}
    其中,干扰项 $I_{u\to k}(\tau)$ 的计算方式与信号功率 $p_{\mathrm{rx}}$ 类似。
\end{itemize}

综合以上两种情况,用户 $k$ 在时刻 $\tau$ 的瞬时数据传输速率(bps)可由香农公式计算得出:
\begin{equation}
 r_k(\tau)=i_s\cdot b\cdot \log_2\big(1+\gamma_k(\tau)\big)
\end{equation}

\subsubsection{任务到达与队列演化}

本部分关于任务到达、队列演化的数学模型与前几问保持一致,但其核心变量——服务量 $S_k(t)$ 的计算,现在深度耦合了第四问新增的跨层接入决策与异构网络干扰环境。

令 $D_k(\tau)\ge 0$ 表示 $1$ms 时刻 $\tau$ 到达用户 $k$ 的任务数据量(Mbit,来自 taskflow 数据)。任务按 FIFO 服务。记窗口起点的队列为 $Q_k(t)$。若窗口 $t$ 内被服务的时隙集合为 $\mathcal{U}_k(t)\subseteq\mathcal{F}(t)$,则窗口内可传输的数据量为
\begin{equation}
 S_k(t)=\sum_{\tau\in\mathcal{U}_k(t)} r_k(\tau)\cdot 10^{-6}\cdot 10^{-3}\quad(\mathrm{Mbit}).
\end{equation}
窗口结束时队列更新为
\begin{equation}
 Q_k(t+100)=\max\Big\{0,\ Q_k(t)+\sum_{\tau\in\mathcal{F}(t)} D_k(\tau)-S_k(t)\Big\}.
\end{equation}

\subsubsection{接入与调度规则}

本问的接入与调度在继承前问规则的基础上,引入了针对异构网络的特定限制:
\begin{itemize}
  \item \textbf{接入限制}:这是本问的核心特征。每个用户 $k$ 的接入选择被严格限定在宏基站(MBS)与地理位置最近的微基站(SBS)之间,即 $a_{n,k}(t)=1$ 仅在 $n=0$ 或 $n=n^*(k)$ 时才可能成立。
  \item \textbf{并发与调度}:各基站(MBS与SBS)的切片并发容量 $C_{n,s}(t)$ 的计算方式,以及窗口内“编号优先、任务完成即补位、URLLC按紧迫度优先”的调度机制,与前文保持一致。
\end{itemize}

\subsubsection{QoS 评估函数}

与前几问一致,服务质量(QoS)的评估基于每个任务的时延和速率表现。

首先,我们需要计算每个任务的总时延。对一个在时刻 $\tau$ 到达的任务,其总时延 $L^{s}_{k,\tau}$ 由排队时延和传输时延构成:
\begin{equation}
    L^{s}_{k,\tau} = (t_{\text{start}} - \tau) + \frac{D_{k,\tau} \cdot 10^6}{\bar{r}_k(t_{\text{start}})}
\end{equation}
其中,$t_{\text{start}}$ 是任务开始服务的时刻,$\bar{r}_k(t_{\text{start}})$ 是服务期间的平均速率。

然后,基于计算出的时延和瞬时速率,我们定义任务级的 QoS 函数 $y^{s}_{k,\tau}$:
\begin{align}
 y^{U}_{k,\tau} &= \begin{cases}
 \alpha^{L^{U}_{k,\tau}} & L^{U}_{k,\tau}\le L^{\text{SLA}}_{U}\\
 -M_U & \text{否则}
 \end{cases} \\
y^{e}_{k,\tau} &= \begin{cases}
 1 & r_{k}(\tau_\text{srv})\ge r^{\text{SLA}}_{e}\ \text{and}\ L^{e}_{k,\tau}\le L^{\text{SLA}}_{e}\\
 \dfrac{r_{k}(\tau_\text{srv})}{r^{\text{SLA}}_{e}} & r_{k}(\tau_\text{srv})< r^{\text{SLA}}_{e}\ \text{and}\ L^{e}_{k,\tau}\le L^{\text{SLA}}_{e}\\
 -M_e & \text{否则}
 \end{cases} \\
 y^{m}_{k,\tau} &= \begin{cases}
 \dfrac{\sum\limits_{i\in\mathcal{K}_m}c'_i}{\sum\limits_{i\in\mathcal{K}_m}c_i} & L^{\,m}_{k,\tau}\le L^{\text{SLA}}_{m}\\
 -M_m & \text{否则}
 \end{cases}
\end{align}
参数取值:$\alpha=0.95$,$r^{\text{SLA}}_e=50\,\mathrm{Mbps}$,$L^{\text{SLA}}_{U}=5\,\mathrm{ms}$,$L^{\text{SLA}}_{e}=100\,\mathrm{ms}$,$L^{\text{SLA}}_{m}=500\,\mathrm{ms}$,$M_U=5, M_e=3, M_m=1$。

\subsubsection{决策变量与优化模型}

决策变量:
\begin{itemize}
  \item RB 切片分配:$x_{n,s}(t)\in\mathbb{Z}_{\ge 0}$
  \item 发射功率:$p_{0,s}(t)\in[10,40]$ (dBm),$p_{n,s}(t)\in[10,30]$ (dBm)
  \item 接入关联:$a_{n,k}(t)\in\{0,1\}$,当 $a_{n,k}(t)=1$ ,则 $k$ 在窗口 $t$ 仅由站 $n$ 调度;若 $n\notin\{0,n^*(k)\}$, $a_{n,k}(t)=0$
\end{itemize}
综合上述要素,第四问的动态联合优化模型可表述为(跨 10 个窗口聚合):
\begin{equation}
\begin{aligned}
\max\limits_{\{x,p,a\}}\quad & Q_{\text{total}}=\sum_{t\in\mathcal{T}}\Bigg[\sum_{k\in\mathcal{K}_U}\sum_{\tau\in\mathcal{A}_k(t)} y^{U}_{k,\tau}+\sum_{k\in\mathcal{K}_e}\sum_{\tau\in\mathcal{A}_k(t)} y^{e}_{k,\tau}+\sum_{k\in\mathcal{K}_m}\sum_{\tau\in\mathcal{A}_k(t)} y^{m}_{k,\tau}\Bigg] \\
\text{s.t.}\quad &
\left\{
\begin{aligned}
& \sum_{s\in\mathcal{S}} x_{n,s}(t)=R_n\\
& x_{n,U}(t)\bmod 10=0,\ x_{n,e}(t)\bmod 5=0,\ x_{n,m}(t)\bmod 2=0\\
& x_{n,s}(t)\in\mathbb{Z}_{\ge 0}\\
& 10\le p_{0,s}(t)\le 40,\ 10\le p_{n,s}(t)\le 30\ (\forall n\in\mathcal{N})\\
& Q_k(t+100)=\max\Big\{0,\ Q_k(t)+\sum_{\tau\in\mathcal{F}(t)} D_k(\tau)-S_k(t)\Big\}\\
& r_k(\tau),\ \gamma_k(\tau)\ \text{由}\ (x,p,a)\ \text{与}\ (\phi,h)\ \text{及调度生成}\\
& \sum_{n\in\bar{\mathcal{N}}} a_{n,k}(t)\le 1\\
& \forall n\in\bar{\mathcal{N}},s\in\mathcal{S},t\in\mathcal{T},\ \forall k,t
\end{aligned}
\right.
\end{aligned}
\end{equation}
其中 $\mathcal{A}_k(t)\subseteq\mathcal{F}(t)$ 为窗口 $t$ 内属于用户 $k$ 且在 SLA 内完成的任务到达时刻集合。\\
该模型体现了“跨层接入选择 + 多站切片 + 切片级功率 + SBS 间互扰 + 任务队列”的耦合,属于时变 MINLP。

\subsection{模型求解}

第四问的优化模型在第三问的基础上引入了宏基站(MBS)与微基站(SBS)的异构结构,并增加了用户接入选择的约束。模型的决策变量维度、干扰关系和资源边界均发生了显著变化,使其成为一个更复杂的混合整数非线性规划(MINLP)问题。直接求解该问题在计算上是不可行的。

我们延续并拓展了问题三中“\textbf{滚动时窗预测控制(MPC)} + \textbf{混合编码遗传算法(GA)}”的求解框架。该框架的总体思想和流程与问题三(见图~\ref{fig:flow_q3})保持一致,即通过 MPC 将长时域问题分解为一系列独立的百毫秒窗口优化问题,再利用 GA 对每个窗口内的“用户接入+资源切片+功率控制”联合优化问题进行高效搜索。本节将重点阐述为适配第四问场景而对 GA 内核所做的关键修改,与问题三重复之处不再赘述。

\subsubsection{内层:混合编码遗传算法 (GA) 的适配}

针对每个决策窗口内的静态资源分配问题,我们对遗传算法的编码、适应度评估和算子进行了如下调整:

\begin{itemize}
    \item \textbf{个体编码方案}:每个个体(染色体)代表一个完整的资源分配策略,其混合编码结构调整如下:
    \begin{enumerate}
        \item \textbf{用户接入决策}:根据模型约束,每个用户在每个窗口只能选择接入宏基站(MBS)或距离其最近的一个微基站(SBS)。因此,该部分编码为一个长度为 48 的\textbf{二进制向量},其中基因位 $i$ 的值为 0 代表用户 $i$ 接入 MBS,为 1 则代表其接入该窗口起始时刻的最近 SBS。这相比问题三的接入决策空间有了大幅简化。
        \item \textbf{RB 切片分配}:编码为一个长度为 $2 \times 4 = 8$ 的整数向量。由于存在 1 个 MBS 和 3 个 SBS,共 4 个基站,我们为每个基站的 URLLC 和 eMBB 切片分配 RB 数量。MBS 的总 RB 数为 100,每个 SBS 为 50。mMTC 切片的 RB 数仍由总数减去 U/E 切片后剩余的资源确定,并向下对齐到切片粒度。
        \item \textbf{切片功率控制}:编码为一个长度为 $4 \times 3 = 12$ 的浮点数向量,分别表示 4 个基站上 3 个切片的发射功率(dBm)。MBS 的功率范围为 $[10, 40]$ dBm,而 SBS 的功率范围为 $[10, 30]$ dBm。
    \end{enumerate}

    \item \textbf{适应度函数}:个体的适应度评估依然通过一个精细的、步长为 1 ms 的窗口仿真器来完成。该仿真器根据解码后的策略进行模拟,但其核心的速率计算模块进行了关键更新:
    \begin{itemize}
        \item \textbf{异构干扰模型}:仿真器精确地实现了第四问的干扰关系。MBS 与 SBS 工作在不同频段,因此 MBS 上的用户不受任何来自 SBS 的干扰。反之,所有 SBS 在同频段工作,因此一个 SBS 上的用户会受到其他所有\textbf{正在服务的、占用同类切片}的 SBS 的同道干扰。
        \item \textbf{动态接入点}:在每个窗口开始时,会根据所有用户当时的坐标预先计算出各自的“最近 SBS”。GA 在解码接入决策时,将基于这一预计算结果来确定用户的具体接入基站。
    \end{itemize}
    最终,仿真器输出的 QoS 总分作为该个体的适应度值。

    \item \textbf{遗传算子与参数}:我们沿用了问题三中的精英保留策略、锦标赛选择、混合交叉(单点交叉用于离散的接入决策,算术交叉用于连续的功率和 RB 变量)以及多模式变异(随机扰动或高斯扰动)。为应对更复杂的问题,参数调整为:种群大小为 50,最大进化代数为 500,交叉概率 0.8,变异概率 0.3。
\end{itemize}

\subsection{结果分析}
