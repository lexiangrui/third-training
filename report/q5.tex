\section{问题五的模型的建立和求解}
\subsection{问题五的描述与分析}

\subsection{预备工作}

\subsection{模型建立}

\subsubsection{能耗模型}

基于附录6的能耗模型,每个基站 $n \in \bar{\mathcal{N}}$ 的总能耗由三部分组成:
\begin{equation}
P_n = P_{static} + P_{RB,n} + P_{tx,n}
\end{equation}

其中:
\begin{itemize}
  \item 固定能耗:$P_{static} = 28\,\mathrm{W}$(基础运行功耗)
  \item RB激活能耗:$P_{RB,n} = \delta \times N_{active,n}$,其中 $\delta = 0.75\,\mathrm{W/RB}$,$N_{active,n} = \sum_{s\in\mathcal{S}} x_{n,s}(t)$ 为基站 $n$ 分配的资源块总数
  \item 发射能耗:$P_{tx,n} = \frac{1}{\eta} P_{transmit,n}$,其中 $\eta = 0.35$ 为效率系数,$P_{transmit,n} = \sum_{s\in\mathcal{S}} p_{n,s}^{(W)}(t) \cdot x_{n,s}(t)$ 为基站 $n$ 的总发射功率
\end{itemize}

功率单位换算:$p_{n,s}^{(W)}(t) = 10^{\frac{p_{n,s}(t)-30}{10}}$,其中 $p_{n,s}(t)$ 为决策变量(单位 dBm)。

全网总能耗为:
\begin{equation}
P_{total} = \sum_{n\in\bar{\mathcal{N}}} P_n
\end{equation}

\subsubsection{多目标优化模型}

在第四问的基础上,增加能耗最小化目标,形成多目标优化问题:
\begin{equation}
\begin{aligned}
\min_{\{x,p,a\}} \quad & \lambda P_{total} - (1-\lambda) Q_{total} \\
\text{s.t.} \quad & \text{第四问的所有约束条件}
\end{aligned}
\end{equation}

其中:
\begin{itemize}
  \item $Q_{total} = \sum_{t\in\mathcal{T}}\sum_{s\in\mathcal{S}}\sum_{k\in\mathcal{K}_s}\sum_{\tau\in\mathcal{A}_k(t)} y^{s}_{k,\tau}$ 为全网用户服务质量总和
  \item $\lambda \in [0,1]$ 为权重系数,调节能耗与服务质量的权衡
  \item 决策变量 $\{x_{n,s}(t), p_{n,s}(t), a_{n,k}(t)\}$ 同第四问
\end{itemize}

\subsection{模型求解}


\subsection{结果分析}

