\section{问题五的模型的建立和求解}
\subsection{问题五的描述与分析}

本问在第四问的异构网络场景和QoS评估框架基础上,引入了基站的能耗模型,旨在探索在保障用户服务质量的同时,实现网络能耗的最小化。这是一个典型的多目标优化问题,要求我们在最大化QoS和最小化能耗这两个相互冲突的目标之间找到最佳平衡。

根据题目“能耗最低的同时能够达到最大的用户服务质量”的描述,我们理解这是一个具有明确优先级顺序的目标:首先应尽可能降低能耗,然后在低能耗的方案中选择能够最大化QoS的策略。为此,我们设计了一个两阶段的优化求解框架。

\subsection{预备工作}

\subsubsection{集合、索引与参数}

本问的集合、索引与大部分参数均与第四问保持一致。
\begin{itemize}
    \item 基站集合:$\bar{\mathcal{N}}=\{0\}\cup\mathcal{N}$,其中 $0$ 表示 MBS,$\mathcal{N}=\{1,2,3\}$ 表示 SBS 集合。
    \item 切片集合:$\mathcal{S}=\{U,e,m\}$。
    \item 用户集合:$\mathcal{K}=\mathcal{K}_U\cup\mathcal{K}_e\cup\mathcal{K}_m$。
    \item 决策时刻集合:$\mathcal{T}=\{0,100,\dots,900\}$ ms。
\end{itemize}

新增的能耗模型相关参数(基于附录6)如下:
\begin{itemize}
    \item 固定功耗:$P_{static} = 28\,\mathrm{W}$。
    \item 每RB激活功耗系数:$\delta = 0.75\,\mathrm{W/RB}$。
    \item 功率放大器效率:$\eta = 0.35$。
\end{itemize}

\subsection{模型建立}

\subsubsection{能耗模型}

基于附录6,每个基站 $n \in \bar{\mathcal{N}}$ 在任意时刻 $\tau$ 的总功耗 $P_n(\tau)$ 由三部分组成:
\begin{equation}
P_n(\tau) = P_{static} + P_{RB,n}(\tau) + P_{tx,n}(\tau)
\end{equation}
其中:
\begin{itemize}
  \item \textbf{固定功耗} $P_{static}$:基站基础运行所需功耗。
  \item \textbf{RB激活功耗} $P_{RB,n}(\tau) = \delta \cdot N_{active,n}(\tau)$,与该时刻激活的RB总数 $N_{active,n}(\tau)$ 成正比。$N_{active,n}(\tau)$ 是基站 $n$ 在时刻 $\tau$ 上服务所有用户的RB数量之和。
  \item \textbf{发射功耗} $P_{tx,n}(\tau) = \frac{1}{\eta} P_{transmit,n}(\tau)$,与总发射功率 $P_{transmit,n}(\tau)$ 成正比。$P_{transmit,n}(\tau)$ 是基站 $n$ 在时刻 $\tau$ 上服务所有用户的发射功率之和,单位为瓦特(W)。
\end{itemize}
全网在时刻 $\tau$ 的总功耗为 $P_{total}(\tau) = \sum_{n\in\bar{\mathcal{N}}} P_n(\tau)$。在一个决策窗口 $t$(时长为 $T_w=100$ms)内的总能耗(单位:焦耳)为:
\begin{equation}
E_{total}(t) = \int_{t}^{t+T_w} P_{total}(\tau) d\tau
\end{equation}

\subsubsection{两阶段优化模型}

为实现“优先最小化能耗,再最大化QoS”的目标,我们将原问题分解为两个 последовательных (sequential) 的子问题,在每个决策窗口 $t \in \mathcal{T}$ 内依次求解。

\paragraph{第一阶段:能耗最小化}
此阶段的目标是找到一组切片功率分配方案 $\{p_{n,s}(t)\}$,使得在固定的接入和RB分配策略下,窗口内的总能耗 $E_{total}(t)$ 最小。
\begin{itemize}
    \item \textbf{固定接入策略}:为简化模型,所有用户均接入距离其最近的基站(MBS或SBS),记为 $a_{n,k}(t) = a_{nearest, k}(t)$。
    \item \textbf{固定RB分配策略}:为保证基础通信能力,各基站的RB资源在不同业务切片间进行均衡分配,记为 $x_{n,s}(t) = x_{equal, n,s}$。
\end{itemize}
该阶段的优化模型表述为:
\begin{equation}
\begin{aligned}
\min_{\{p_{n,s}(t)\}} \quad & E_{total}(t; \{p_{n,s}(t)\}) \\
\text{s.t.} \quad & p_{min, n} \le p_{n,s}(t) \le p_{max, n}, \quad \forall n \in \bar{\mathcal{N}}, s \in \mathcal{S}
\end{aligned}
\end{equation}
决策变量仅为各基站各切片的发射功率 $p_{n,s}(t)$。

\paragraph{第二阶段:QoS最大化}
在第一阶段得到的最优功率分配方案 $p^*_{n,s}(t)$ 的基础上,此阶段通过优化RB切片分配 $\{x_{n,s}(t)\}$ 来最大化全网用户的总服务质量 $Q_{total}(t)$。
\begin{itemize}
    \item \textbf{固定功率策略}:采用第一阶段优化得到的功率 $p_{n,s}(t) = p^*_{n,s}(t)$。
    \item \textbf{固定接入策略}:同样为最近基站接入 $a_{n,k}(t) = a_{nearest, k}(t)$。
\end{itemize}
该阶段的优化模型表述为:
\begin{equation}
\begin{aligned}
\max_{\{x_{n,s}(t)\}} \quad & Q_{total}(t; \{x_{n,s}(t)\}) \\
\text{s.t. 第四问的所有约束条件}  
\end{aligned}
\end{equation}
决策变量为各基站为不同业务切片分配的RB数量 $x_{n,s}(t)$,它必须满足总数约束和相应切片的粒度约束。

\subsection{模型求解}

我们延续“滚动时域控制(MPC)”框架,在每个100ms的窗口内,依次执行上述两阶段优化。

\paragraph{第一阶段求解:遗传算法 (GA)}
由于能耗与功率之间存在复杂的非线性关系,且其评估依赖于精细的动态仿真,我们采用遗传算法(GA)求解第一阶段的功率优化问题。
\begin{itemize}
    \item \textbf{编码}:个体仅编码各基站各切片的功率值,为一个长度为 $4 \times 3 = 12$ 的浮点数数组。
    \item \textbf{适应度函数}:适应度被定义为窗口内总能耗的负值,即 $f = -E_{total}$。该能耗值通过调用一个1ms步长的仿真器,在固定的接入和RB分配下模拟整个窗口得出。
    \item \textbf{遗传算子}:采用锦标赛选择、算术交叉和高斯扰动变异。
\end{itemize}

\paragraph{第二阶段求解:枚举}
该阶段方法与第二问大致相同,区别在于第四问开始增加了宏基站,在枚举时需要同时考虑宏基站和微基站的RB分配。

\subsection{结果分析}

