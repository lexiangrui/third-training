\section{问题二的模型的建立和求解}
\subsection{问题二的描述与分析}

问题二考虑动态环境下的多时间段资源分配场景。与问题一的静态单时刻分配不同,问题二面临用户移动性、信道时变性以及任务队列动态变化的复杂问题。在1000ms的观测窗口内,系统需要每100ms进行一次资源分配决策(共10次),既要处理新到达的任务,又要考虑积压在排队队列中的历史任务。这是一个多阶段动态优化问题,需要在时间维度上综合考虑任务到达、信道变化和排队延迟的耦合影响。

\subsection{预备工作}

\subsection{模型建立}

问题二模型大部分与问题一相同,不同之处在于问题二允许 $t$ 在决策窗口内按 $\mathcal{T}=\{0,100,\dots,900\}$ 演化,并引入任务到达与队列动态。为避免重复,问题二中与问题一中相同的定义与公式不再赘述。

\subsubsection{任务队列动态演化模型}

为精确描述任务的动态变化,我们引入两个时间尺度:
\begin{itemize}
    \item \textbf{决策时刻 $t$}:每100ms进行一次资源分配决策,对应$t \in \{0, 100, 200, \ldots, 900\}$。
    \item \textbf{仿真时刻 $\tau$}:以1ms为步长,用于模拟任务到达和信道变化,$\tau$表示具体的毫秒时刻。
\end{itemize}

定义用户$k$在时刻$t$的任务队列状态:

\begin{itemize}
  \item $A_{k,\tau}(t)$:用户$k$在时刻$\tau$到达且在时刻$t$仍在队列中的任务数据量(Mbit)
  \item $Q_k(t) = \sum_{\tau=0}^{t} A_{k,\tau}(t)$:用户$k$在时刻$t$的总排队任务量
  \item $W_k(t)$:用户$k$在时刻$t$已等待的排队时间
\end{itemize}

任务队列的动态演化遵循以下规律:

\textbf{(1) 任务到达:}
在每个1ms时刻$\tau$,根据数据文件读取新到达任务:
\begin{equation}
A_{k,\tau}(\tau) = \text{TaskFlow}_k(\tau)
\end{equation}

\textbf{(2) 任务服务:}
在决策时刻$t$,若用户$k$被分配$i_k(t)$个RB,则在接下来的100ms内可传输的数据量为:
\begin{equation}
S_k(t) = r_k(t) \times 0.1 \times 10^{-6} \quad \text{(Mbit)}
\end{equation}

\textbf{(3) 队列更新:}
任务按FIFO(先进先出)顺序服务,队列更新规则为:
\begin{equation}
\label{eq:queue_evolution}
Q_k(t+100) = \max\left(0, Q_k(t) + \sum_{\tau=t+1}^{t+100} \text{TaskFlow}_k(\tau) - S_k(t)\right)
\end{equation}

\subsubsection{时延计算模型(任务级)}
 
由于存在到达过程,问题二按任务到达时刻进行时延度量。对于用户$k$在时刻$\tau$到达的任务,其总时延包含排队时延和传输时延:

\textbf{(1) 排队时延:}
任务在时刻$\tau$到达,在时刻$t_{\text{start}}$开始服务,则排队时延为:
\begin{equation}
Q_{k,\tau} = t_{\text{start}} - \tau
\end{equation}

\textbf{(2) 传输时延:}
假设任务从时刻$t_{\text{start}}$开始传输,数据量为$D_{k,\tau}$,则传输时延为:
\begin{equation}
T_{k,\tau} = \frac{D_{k,\tau} \times 10^6}{r_k(t_{\text{start}})}
\end{equation}

\textbf{(3) 总时延:}
\begin{equation}
L_{k,\tau}^s = Q_{k,\tau} + T_{k,\tau}, \quad s \in \{U, e, m\}
\end{equation}

\subsubsection{动态服务质量评估函数}
 
在多时间段场景下,各切片的QoS评估在问题一基础上作如下时间聚合:

\textbf{(1) URLLC切片:}
对于在时间窗口$[0, 1000]$ms内完成的所有URLLC任务:
\begin{equation}
y_{k,\tau}^{U} = \begin{cases}
\alpha^{L_{k,\tau}^{U}} & \text{若 } L_{k,\tau}^{U} \leq L_{U}^{\text{SLA}} \\
-M_{U} & \text{若 } L_{k,\tau}^{U} > L_{U}^{\text{SLA}}
\end{cases}
\end{equation}

\textbf{(2) eMBB切片:}
对于在决策时刻$t$服务的eMBB用户$k$:
\begin{equation}
y_{k}^{e}(t) = \begin{cases}
1 & \text{若 } r_k(t) \geq r_{e}^{\text{SLA}} \text{ 且 } L_{k,\tau}^{e} \leq L_{e}^{\text{SLA}} \\
\frac{r_k(t)}{r_{e}^{\text{SLA}}} & \text{若 } r_k(t) < r_{e}^{\text{SLA}} \text{ 且 } L_{k,\tau}^{e} \leq L_{e}^{\text{SLA}} \\
-M_{e} & \text{若} L_{k,\tau}^{e} > L_{e}^{\text{SLA}}
\end{cases}
\end{equation}

\textbf{(3) mMTC切片:}
在每个决策时刻$t$,对每个mMTC用户$k$的QoS评估为:
\begin{equation}
y_k^{m}(t) = \begin{cases}
\dfrac{\sum_{i \in \mathcal{U}_{m}} c_i'(t)}{\sum_{i \in \mathcal{U}_{m}} c_i(t)} & \text{若 } L_k^{m}(t) \le L_{m}^{\text{SLA}} \\
-M_{m} & \text{若 } L_k^{m}(t) > L_{m}^{\text{SLA}}
\end{cases}
\end{equation}
其中,$c_i(t)$表示用户$i$在时刻$t$是否有待服务任务,$c_i'(t)$表示是否成功接入服务。

\subsubsection{多时间段优化模型}
 
基于上述分析,建立如下多时间段动态优化模型(目标在整个时间窗口跨$t$求和):

\begin{equation}
\begin{aligned}
\max_{\{n_s(t)\}_{s,t}} \quad & Q_{\text{total}} = \sum_{t \in \mathcal{T}} \left[ \sum_{k \in \mathcal{U}_U} y_{k}^{U}(t) + \sum_{k \in \mathcal{U}_e} y_{k}^{e}(t) + \sum_{k \in \mathcal{U}_m} y_{k}^{m}(t) \right]\\
\text{s.t.} \quad & \begin{cases}
 n_U(t) + n_e(t) + n_m(t) = 50\\
 n_U(t) \bmod 10 = 0,\; n_e(t) \bmod 5 = 0,\; n_m(t) \bmod 2 = 0\\
 Q_k(t+100) = \max\!\left(0, Q_k(t) + \Delta A_k(t) - S_k(t)\right) \\
 n_s(t) \in \mathbb{Z}_{\ge 0}\\
 \forall t \in \mathcal{T},\;s \in \mathcal{S},\; \forall k
 \end{cases}
 \end{aligned}
 \end{equation}
其中,$\mathcal{T} = \{0, 100, 200, \ldots, 900\}$为决策时刻集合,$n_s(t)$为时刻$t$分配给切片$s$的RB数量,$Q_k(t)$为用户$k$在时刻$t$的排队任务量,$\Delta A_k(t)$为时间段$[t, t+100)$内用户$k$的新增任务量。

该模型的核心挑战在于:(1) 状态空间的指数级增长;(2) 任务到达的随机性与信道的时变性;(3) 排队时延与传输时延的耦合优化。需要设计高效的求解算法来处理这一复杂的多阶段随机优化问题。

\subsection{模型求解}

问题二是一个多阶段动态优化问题。由于状态空间(用户队列、信道状况)随时间演化,且任务到达具有随机性,精确求解该问题通常需要复杂的动态规划或强化学习方法,计算成本极高。考虑到决策周期较短,且需要快速响应,我们采用一种基于模型预测控制(Model Predictive Control, MPC)思想的贪心策略(Myopic Policy),也称为“单步前瞻优化”。该策略在每个决策窗口的起点,仅优化当前窗口的性能,而不考虑对未来窗口的长期影响。这种方法在实践中被证明是有效且计算可行的。

算法的核心思想是:在每个决策时刻 $t \in \{0, 100, \dots, 900\}$,我们面对当前的系统状态(主要是各用户的任务队列),通过枚举所有可能的RB分配方案,并对每种方案进行精细化的仿真,来预测未来100ms内的系统服务质量。然后,我们选择能使当前窗口QoS最大化的方案进行实施,并将演化后的系统状态作为下一个决策时刻的初始条件。具体算法流程如下:

\textbf{Step1:初始化}

在仿真开始时刻 $t=0$,初始化所有用户的任务队列为空。

\textbf{Step2:逐窗口迭代决策}

对于每个决策窗口 $w$(对应时间段 $[t, t+100)$,其中 $t = w \times 100$):

\begin{enumerate}
    \item \textbf{生成RB分配方案}:与问题一类似,我们首先枚举所有满足约束条件的RB分配方案 $(n_U(t), n_e(t), n_m(t))$。确保资源总量为50,且每个切片的RB数量满足其最小占用粒度(URLLC为10的倍数,eMBB为5的倍数,mMTC为2的倍数)。

    \item \textbf{窗口内调度仿真与评估}:对于每一个生成的RB分配方案,我们执行一次为期100ms的详细仿真,以评估其性能。
    \begin{itemize}
        \item \textbf{初始状态}:仿真从当前决策时刻 $t$ 的系统状态开始,包括所有用户当前的排队任务。
        \item \textbf{动态演化}:在100ms的仿真窗口内,我们以1ms为步长进行演化:
        \begin{itemize}
            \item \textbf{任务到达}:根据附件二的数据,在每个1ms时刻,将新到达的任务加入对应用户的队列。
            \item \textbf{调度与服务}:根据当前切片的并发容量 $C_s = \lfloor n_s(t) / v_s \rfloor$,采用“编号靠前优先”的原则,为有任务排队的用户分配服务信道。正在服务的用户将根据其在当前ms的信道质量计算出的速率来处理任务。
            \item \textbf{资源释放与接续}:当一个用户的任务完成时,其占用的并发槽位被立即释放。调度器会立刻检查该切片内是否有其他排队的用户(同样按编号顺序),若有则立即接续服务。
        \end{itemize}
        \item \textbf{性能计算}:在仿真过程中,我们精确记录每个任务的到达、开始服务和完成的时刻,从而计算其端到端时延。对于在当前100ms窗口内完成的每一个任务,我们根据其所属切片的服务质量评估函数计算QoS得分。
    \end{itemize}
    
    \item \textbf{选择最优方案并更新状态}:遍历所有RB分配方案后,我们选择在当前窗口内获得累计QoS总分最高的方案作为本次决策的结果。然后,我们将该最优方案对应的仿真结束时刻($t+100$)的用户队列状态,作为下一个决策窗口的初始状态。
\end{enumerate}

\textbf{Step3:汇总结果}

重复Step2,直到完成所有10个决策窗口($t=0$至$t=900$)的决策。最后,将每个窗口获得的最优QoS得分进行累加,得到整个1000ms内的总服务质量。通过这种方式,我们得到了一系列动态的RB分配决策,以及最终的系统整体性能评估。

\subsection{结果分析}

通过执行上述基于MPC的贪心算法,我们得到了1000ms内的动态资源分配策略,其总服务质量达到了\textbf{352.1029}。

\subsubsection{最优资源分配序列}

算法在10个决策窗口中选择的RB分配序列如下表所示。该序列是在每个窗口选择瞬时最优解(若有多个则选择第一个)的结果。

\begin{table}[H]
\centering
\caption{问题二动态资源分配序列}
\label{tab:q2_decision_sequence}
\begin{tabular}{ccccc}
\hline
\textbf{决策时刻 (ms)} & \textbf{URLLC RB数 ($n_U$)} & \textbf{eMBB RB数 ($n_e$)} & \textbf{mMTC RB数 ($n_m$)} & \textbf{窗口QoS} \\
\hline
0 & 10 & 20 & 20 & 65.490 \\
100 & 10 & 20 & 20 & 40.028 \\
200 & 10 & 20 & 20 & 38.835 \\
300 & 10 & 0 & 40 & 32.800 \\
400 & 20 & 0 & 30 & 31.850 \\
500 & 10 & 0 & 40 & 24.250 \\
600 & 10 & 0 & 40 & 27.100 \\
700 & 20 & 0 & 30 & 32.800 \\
800 & 20 & 0 & 30 & 31.850 \\
900 & 20 & 0 & 30 & 27.100 \\
\hline
\textbf{总计} & - & - & - & \textbf{352.103} \\
\hline
\end{tabular}
\end{table}

在整个仿真周期内,各切片累计获得的QoS分数分别为:URLLC QoS合计\textbf{205.8175},eMBB QoS合计\textbf{46.2854},mMTC QoS合计\textbf{100.0000}。

\subsubsection{并列最优解分析}

值得注意的是,在多个决策窗口中,算法发现了多个能够达到相同最优QoS的RB分配方案。这为网络运营商提供了在满足性能目标的同时,根据其他策略(如能耗、资源均衡等)进行选择的灵活性。下表列出了部分存在并列最优解的决策时刻及其方案。

\begin{table}[H]
\centering
\caption{部分决策窗口的并列最优方案}
\label{tab:q2_tie_solutions}
\begin{tabular}{ccc}
\hline
\textbf{决策时刻 (ms)} & \textbf{窗口最优QoS} & \textbf{并列最优RB分配方案 ($n_U, n_e, n_m$)} \\
\hline
\multirow{3}{*}{300} & \multirow{3}{*}{32.800} & (10, 0, 40) \\
 & & (20, 0, 30) \\
 & & (30, 0, 20) \\
\hline
\multirow{3}{*}{400} & \multirow{3}{*}{31.850} & (20, 0, 30) \\
 & & (30, 0, 20) \\
 & & (40, 0, 10) \\
\hline
\multirow{3}{*}{600} & \multirow{3}{*}{27.100} & (10, 0, 40) \\
 & & (20, 0, 30) \\
 & & (30, 0, 20) \\
\hline
\multirow{2}{*}{900} & \multirow{2}{*}{27.100} & (20, 0, 30) \\
 & & (30, 0, 20) \\
\hline
\end{tabular}
\end{table}

\begin{itemize}
    \item \textbf{动态适应性}:从资源分配序列(表 \ref{tab:q2_decision_sequence})可以看出,算法展现了良好的动态适应性。在仿真初期(0-300ms),eMBB业务有大量任务到达,算法明智地为其分配了20个RB以快速处理,获得了较高的QoS。在300ms后,eMBB任务队列清空,算法果断地将其RB资源完全回收,转而分配给URLLC和mMTC,以应对后续的URLLC任务并保证mMTC的稳定接入。
    \item \textbf{资源分配的灵活性}:如表 \ref{tab:q2_tie_solutions} 所示,从300ms开始,系统中出现了大量的并列最优解。这些方案的共同点是都放弃了为eMBB分配资源(因为其队列已空),而将资源在URLLC和mMTC之间进行权衡。例如在300ms时,(10, 0, 40), (20, 0, 30), (30, 0, 20)三种方案都能达到相同的最优QoS。这表明在满足核心性能指标后,资源配置具有相当的灵活性。运营商可以根据长期策略选择更偏向URLLC(保障未来可能的高优先级任务)或更偏向mMTC(扩大连接容量)的方案。
    \item \textbf{切片性能分析}:URLLC切片的所有任务均在远低于SLA(5ms)的时延内完成,获得了很高的QoS分数。eMBB切片在前期获得了足够的资源,其QoS贡献主要来自前300ms。mMTC切片由于其“尽力而为”和高并发的特性,在整个周期内稳定地获得了满分的QoS。这说明我们采用的MPC贪心策略成功地平衡了三类业务的需求。
\end{itemize}

综上所述,我们提出的动态资源分配模型与求解算法,能够在时变的信道和任务到达条件下,做出快速有效的决策,实现了系统在整个时间窗口内总服务质量的最优。