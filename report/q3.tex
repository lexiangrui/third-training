\section{问题三的模型的建立和求解}
\subsection{问题三的描述与分析}

\subsection{预备工作}

\subsection{模型建立}

\subsubsection{集合、索引与参数}

为适配附件三的多微基站、同频复用且存在小区间干扰的场景,定义如下集合与索引:

\begin{itemize}
  \item 基站集合:$\mathcal{N}=\{1,2,3\}$(分别对应 BS1、BS2、BS3)。
  \item 切片集合:$\mathcal{S}=\{U,e,m\}$,分别对应 URLLC、eMBB、mMTC。
  \item 用户集合:$\mathcal{K}=\mathcal{K}_U\cup\mathcal{K}_e\cup\mathcal{K}_m$,其中 $\mathcal{K}_U=\{\mathrm{U1},\mathrm{U2}\}$,$\mathcal{K}_e=\{\mathrm{e1},\dots,\mathrm{e12}\}$,$\mathcal{K}_m=\{\mathrm{m1},\dots,\mathrm{m30}\}$。
  \item 决策时刻集合:$\mathcal{T}=\{0,100,\dots,900\}$(单位 ms),每个决策窗口长度为 $100$ ms;窗口内以 $1$ ms 步长进行链路与队列仿真,记窗口内细粒度时刻集合为 $\mathcal{F}(t)=\{t,t+1,\dots,t+99\}$。
\end{itemize}

关键系统参数(与问题一、二保持一致):

\begin{itemize}
  \item 每站可用 RB 总数 $R_{\text{tot}}=50$;单 RB 带宽 $b=360\,\mathrm{kHz}$;噪声系数 $NF=7\,\mathrm{dB}$;
  \item 切片占用粒度:$i_U=10,\ i_e=5,\ i_m=2$(每个并发用户占用的 RB 数);
  \item 发射功率决策范围(第三问):$p_{n,s}(t)\in[10,30]\,\mathrm{dBm}$,为“基站 $n$ 在窗口 $t$ 对切片 $s$ 的统一每 RB 功率”(切片内各 RB 功率一致)。
\end{itemize}

\subsubsection{信道与干扰模型}

附件三提供了每 $1$ ms 的大规模损耗 $\phi_{n,k}(\tau)$(dB)与小规模瑞利衰落 $h_{n,k}(\tau)$。设窗口 $t\in\mathcal{T}$ 内细粒度时刻为 $\tau\in\mathcal{F}(t)$,若用户 $k\in\mathcal{K}_s$ 在窗口 $t$ 由基站 $n$ 服务且被分配切片 $s$ 的 RB,则其接收功率(mW)为
\begin{equation}
 p_{\mathrm{rx},n\to k}(\tau)=10^{\frac{p_{n,s}(t)-\phi_{n,k}(\tau)}{10}}\cdot |h_{n,k}(\tau)|^2.
\end{equation}

噪声功率与占用 RB 数 $i$ 成正比,换算为线性功率(mW):
\begin{equation}
 N_0(i)=10^{\frac{-174+10\log_{10}(i\cdot b)+NF-30}{10}}.
\end{equation}

微基站同频复用引入同信道干扰。为保持“同一 RB 索引才互扰”的规则,我们令每站在窗口 $t$ 内将其 $50$ 个 RB 在频域上按切片连续划分且次序固定(例如 U-e-m),每个切片获得一段连续 RB 区间,跨站的相同 RB 索引构成同信道。于是用户 $k$ 的瞬时信干噪比为
\begin{equation}
 \gamma_k(\tau)=\frac{p_{\mathrm{rx},n\to k}(\tau)}{\sum\limits_{u\in\mathcal{N},\ u\neq n} I_{u\to k}(\tau)+N_0(i_s)},\quad s\in\mathcal{S},
\end{equation}
其中 $I_{u\to k}(\tau)$ 表示来自他站 $u$、在与 $k$ 所占 RB 索引重叠的切片 RB 上的干扰功率,按与上式相同的接收功率表达(由 $p_{u,s'}(t),\phi_{u,k}(\tau),h_{u,k}(\tau)$ 决定)。基于香农公式,窗口内瞬时速率为
\begin{equation}
 r_k(\tau)=i_s\cdot b\cdot \log_2\big(1+\gamma_k(\tau)\big)\quad(\mathrm{bps}).
\end{equation}

\subsubsection{任务到达与队列演化}

令 $D_{k}(\tau)\ge 0$ 表示 $1$ ms 时刻 $\tau$ 到达用户 $k$ 的任务数据量(Mbit,来自 taskflow 数据)。任务按 FIFO 服务。记 $Q_k(t)$ 为窗口起点 $t$ 的总排队量(Mbit)。若窗口 $t$ 内被服务的时隙集合为 $\mathcal{U}_k(t)\subseteq\mathcal{F}(t)$,则窗口内可传输的数据量为
\begin{equation}
 S_k(t)=\sum_{\tau\in\mathcal{U}_k(t)} r_k(\tau)\cdot 10^{-6}\cdot 10^{-3}\quad(\mathrm{Mbit}),
\end{equation}
其中 $10^{-3}$ 将 $\mathrm{bps}$ 与 $1$ ms 时长换算为 bit,再以 $10^{-6}$ 换算为 Mbit。窗口结束时队列更新为
\begin{equation}
 Q_k(t+100)=\max\Big\{0,\ Q_k(t)+\sum_{\tau\in\mathcal{F}(t)} D_k(\tau)-S_k(t)\Big\}.
\end{equation}

\subsubsection{RB 切片并发与调度规则}

令 $x_{n,s}(t)$ 为窗口 $t$ 时基站 $n$ 切片 $s$ 的 RB 数,满足 $\sum_{s\in\mathcal{S}}x_{n,s}(t)=R_{\text{tot}}$。切片 $s$ 的并发容量为
\begin{equation}
 C_{n,s}(t)=\big\lfloor x_{n,s}(t)/i_s\big\rfloor.
\end{equation}
窗口内采用“编号靠前优先”的串-并行调度:每个 $(n,s)$ 最多同时服务 $C_{n,s}(t)$ 个队头任务,任务完成即释放并补位。URLLC 在切片内可按紧迫度(距 SLA 的剩余时限)优先。

\subsubsection{QoS 评估函数}

与问题一、二一致,定义任务级 QoS:
\begin{equation}
 y^{U}_{k,\tau}=\begin{cases}
 \alpha^{L^{U}_{k,\tau}} & L^{U}_{k,\tau}\le L^{\text{SLA}}_{U}\\
 -M_U & \text{否则}
 \end{cases},\quad
 y^{e}_{k,\tau}=\begin{cases}
 1 & r_{k}(\tau_\text{srv})\ge r^{\text{SLA}}_{e}\ \&\ L^{e}_{k,\tau}\le L^{\text{SLA}}_{e}\\
 \dfrac{r_{k}(\tau_\text{srv})}{r^{\text{SLA}}_{e}} & r_{k}(\tau_\text{srv})< r^{\text{SLA}}_{e}\ \&\ L^{e}_{k,\tau}\le L^{\text{SLA}}_{e}\\
 -M_e & \text{否则}
 \end{cases}
\end{equation}
\begin{equation}
 y^{m}_{k,\tau}=\begin{cases}
 \dfrac{\sum\limits_{i\in\mathcal{K}_m}c'_i}{\sum\limits_{i\in\mathcal{K}_m}c_i} & L^{\,m}_{k,\tau}\le L^{\text{SLA}}_{m}\\
 -M_m & \text{否则}
 \end{cases}
\end{equation}
其中 $L^{s}_{k,\tau}=Q_{k,\tau}+T_{k,\tau}$ 为任务总时延,$\tau_\text{srv}$ 为其被服务时段代表点;$c_i,c'_i$ 分别表示 mMTC 的“有任务/成功接入”标记。SLA 与惩罚参数同问题一中表述:$\alpha=0.95$,$r^{\text{SLA}}_e=50\,\mathrm{Mbps}$,$L^{\text{SLA}}_{U}=5\,\mathrm{ms}$,$L^{\text{SLA}}_{e}=100\,\mathrm{ms}$,$L^{\text{SLA}}_{m}=500\,\mathrm{ms}$,$M_U=5, M_e=3, M_m=1$。

\subsubsection{决策变量与优化模型}

决策变量:
\begin{itemize}
  \item RB 切片分配:$x_{n,s}(t)\in\mathbb{Z}_{\ge 0}$;
  \item 发射功率:$p_{n,s}(t)\in[10,30]\,\mathrm{dBm}$;
  \item 接入关联:$a_{n,k}(t)\in\{0,1\}$,$\sum\limits_{n\in\mathcal{N}}a_{n,k}(t)\le 1$;若 $a_{n,k}(t)=1$ 则 $k$ 在窗口 $t$ 仅由站 $n$ 调度。
\end{itemize}

综合上述要素,第三问的动态联合优化模型可表述为(跨 10 个窗口聚合):
\begin{equation}
\begin{aligned}
\max_{\{x,p,a\}}\quad & Q_{\text{total}}=\sum_{t\in\mathcal{T}}\Bigg[\sum_{k\in\mathcal{K}_U}\sum_{\tau\in\mathcal{A}_k(t)} y^{U}_{k,\tau}+\sum_{k\in\mathcal{K}_e}\sum_{\tau\in\mathcal{A}_k(t)} y^{e}_{k,\tau}+\sum_{k\in\mathcal{K}_m}\sum_{\tau\in\mathcal{A}_k(t)} y^{m}_{k,\tau}\Bigg] \\
\text{s.t.}\quad & 
\left\{
\begin{aligned}
& \sum_{s\in\mathcal{S}} x_{n,s}(t)=50,\ \forall n\in\mathcal{N},\ t\in\mathcal{T}, \\
& x_{n,U}(t)\bmod 10=0,\ x_{n,e}(t)\bmod 5=0,\ x_{n,m}(t)\bmod 2=0, \\
& x_{n,s}(t)\in\mathbb{Z}_{\ge 0},\ \forall n,s,t, \\
& 10\,\mathrm{dBm}\le p_{n,s}(t)\le 30\,\mathrm{dBm},\ \forall n,s,t, \\
& \sum_{n\in\mathcal{N}} a_{n,k}(t)\le 1,\ a_{n,k}(t)\in\{0,1\},\ \forall k,t, \\
& r_k(\tau),\ \gamma_k(\tau)\ \text{由}\ (x,p,a)\ \text{与}\ (\phi,h)\ \text{及调度生成}, \\
& Q_k(t+100)=\max\Big\{0,\ Q_k(t)+\sum_{\tau\in\mathcal{F}(t)} D_k(\tau)-S_k(t)\Big\},\ \forall k,t.
\end{aligned}
\right.
\end{aligned}
\end{equation}
其中 $\mathcal{A}_k(t)\subseteq\mathcal{F}(t)$ 为窗口 $t$ 内属于用户 $k$ 且在 SLA 内完成的任务到达时刻集合;$r_k(\tau)$ 与 $S_k(t)$ 均受干扰耦合与调度影响,是 $(x,p,a)$ 的非线性函数。该模型体现了“多站同频干扰 + 切片化 RB 分配 + 切片级功率控制 + 任务队列”的耦合特性,属于带整数约束与非凸干扰项的时变 MINLP 问题。

\subsection{模型求解}

\textbf{Step1:} 

\textbf{Step2:} 

\textbf{Step3:} 

\subsection{求解结果}